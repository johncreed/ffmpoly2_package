\section{Related work}
Our research provides a novel perspective of feature interaction and improves the CTR prediction. The most related domains are focused learning and feature interaction related models for CTR prediction. We will review studies of these two domains.

One closely related domain is a novel learning regime, named focused learning \cite{Beutel:2017:BGO:3038912.3052713} proposed by Google. To solve the ill-served items in recommendation, they divide the set of items into different subsets according to different objectives. They further conduct many experiments to demonstrate the effectiveness. Other related works include that some learn local models \cite{Christakopoulou:2016:LIM:2959100.2959185}, and some ensemble local learners \cite{Beutel:2015:AAC:2736277.2741091}. Our study is inspired by focused learning, but we focus on different ways for adaptively modeling feature conjunction.

CTR prediction plays an important role in recommendation. Many machine learning models have been proposed for this task. Google \cite{McMahan:2013:ACP:2487575.2488200} reported that their CTR prediction system is based on an online learning algorithm to learn a linear model. Their algorithm follows from the method follow-the-regularized-leader (\text{FTRL}) \cite{pmlr-v15-mcmahan11b}. Unlike a linear model, Facebook \cite{He:2014:PLP:2648584.2648589} provides a tree-based model for their CTR prediction system, in which the tree model is served as a way for feature engineering. Many CTR prediction models are proposed based on FM \cite{5694074}. We briefly discuss some of them. A tensor based model \cite{Rendle:2010:PIT:1718487.1718498} is used in personalized tag recommendation. FFM \cite{Juan:2016:FFM:2959100.2959134} has been proven to be an effective model in real CTR advertising system \cite{Juan:2017:FFM:3041021.3054185}. Difacto \cite{Li:2016:DDF:2835776.2835781} is the first one that expands FM to large-scale machine learning systems. Instead of using a fixed $k$ as the length of the latent vector, Difacto introduces the memory adaptive constraint, making the value $k$ depend on a single feature frequency. In terms of feature-pair interactions, AFM \cite{ijcai2017-435} is proposed to solve the weights of feature interactions via attentional networks. Starting from a bad case in FM, \cite{Prillo:2017:EVF:3109859.3109892} provides an elementary view on FM and also discusses issues of feature conjunction in FM. With the deep learning method, DeepFM \cite{Guo:2017:DFB:3172077.3172127} learns more sophisticated feature interactions. Different from these studies, our data adaptive model provides a novel insight on feature conjunction. The proposed model shows an improved CTR prediction on different real data sets.
